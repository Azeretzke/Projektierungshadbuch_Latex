\chapter{Projektcheckliste}\label{chap:Projektcheckliste}

    \begin{longtable}{| p{\colwidth{0.12}} | p{\colwidth{0.8}} | p{\colwidth{0.08}} |} % columns widths have to add up to 1 otherwise there will be an over- or underfill of the table layout
        \hline
        Typ & Punkt & OK \\
        \hline
        \endhead % save header to repeat it on each page          
        Projekt & Editiersprache und Referenzsprache korrekt eingestellt &  \\
        \hline      
        Projekt & \glqq Beim Übersetzen von Bausteinen Simulierbarkeit unterstützen\grqq{} aktiviert &  \\
        \hline      
        HW & SPS-Rechenleistung ausreichend &  \\
        \hline
        HW & SPS-Passwörter und Zugriffsschutz eingestellt &  \\
        \hline
        HW & SPS: Mehrsprachigkeit korrekt eingestellt &  \\
        \hline
        HW & SPS: F-Destination- und F-Source-Adressbereiche korrekt vergeben (Unter-/Obergrenze für F-Zieladressen, Zentrale F-Quelladresse) &  \\
        \hline
        HW & SPS Standard F-Überwachungszeit auf mindestens 300 ms &  \\
        \hline
        HW & Netzwerkteilnehmer Profisafe-Adressen Profisafe-Adresstyp 1 \cite[67ff]{Siemens:Profisafe} korrekt eingestellt(SPS-Adressbereiche beachten) &  \\
        \hline
        HW & An allen Netzwerkschnittstellen Default F-Überwachungszeit auf 300ms eingestellt &  \\
        \hline
        HW & Netzwerkteilnehmer vollständig angelegt (ET-Stationen, Antriebe, …) &  \\
        \hline
        HW & EA-Adressbereiche richtig eingestellt &  \\
        \hline
        HW & IP-Adressen eingestellt &  \\
        \hline
        HW & Teilnehmernummern entsprechend IP-Adresse eingestellt &  \\
        \hline
        HW & F-Karten Kanäle korrekt parametriert (ein-/zwei-kanalige Auswertung, Ka-näle deaktiviert/aktiviert, Spannungsversorgung, Diskrepanzzeit, …) &  \\
        \hline
        HW & Analogkarten korrekt parametriert (0-10V / 0-20mA / 4-20mA / RTD, …, Kanäle deaktiviert/aktiviert) &  \\
        \hline
        HW & Identification \& Maintenance ausgefüllt &  \\
        \hline
        HW & Baugruppen kommentiert &  \\
        \hline
        HW & Profinet-Gerätenamen entsprechend Norm vergeben &  \\
        \hline
        HW & SPS Uhrzeitsynchronisation aktiviert (falls NTP-Server vorhanden) &  \\
        \hline
        HW & Betriebsart IO-Device aktiviert und parametriert (falls nötig) &  \\
        \hline
        HW & Netzwerktopologie projektiert  &  \\
        \hline
        HW & Sync-Domäne konfiguriert (zwingend erforderlich bei IRT, z.B. Time-Based-IO) &  \\
        \hline
        HW & Redundanz-Domäne konfiguriert (falls gefordert) &  \\
        \hline
        HW & SPS Webserverzugriff aktiviert / deaktiviert (je nach Kundenwunsch) und konfiguriert &  \\
        \hline
        HW & SPS Anlauf nach NETZ-Ein auf Warmstart – RUN eingestellt &  \\
        \hline
        HW & SPS Vergleich Sollausbau zu Istausbau auf Anlauf der CPU auch bei Unter-schieden eingestellt &  \\
        \hline
        HW & SPS maximale Zykluszeit auf mindestens 500 ms eingestellt &  \\
        \hline
        HW & System- und Taktmerker aktiviert &  \\
        \hline
        HW & SPS PLC-Meldungen Zentrale Meldungsverwaltung in der PLC aktiviert &  \\
        \hline
        HW & SPS Uhrzeit eingestellt (Zeitzone etc.) &  \\
        \hline
        HW & SPS Stromversorgung parametriert &  \\
        \hline
        SW & Meldetextlisten für S120 / G120 eingefügt &  \\
        \hline
        SW & EA-Symbolik vollständig &  \\
        \hline
        SW & Kein Zugriff auf nicht in der Hardware parametrierte Adressen 
        (Software Gesamtübersetzen -> Warnungen kontrollieren -> Warnung „Es wird auf nicht projektierte Ein-/Ausgänge zugegriffen“ darf nicht erscheinen)
         &  \\
        \hline
        SW & F-Programm ohne Warnungen &  \\
        \hline
        SW & F-Programmzyklus auf 100ms eingestellt &  \\
        \hline
        SW & Systemdiagnose programmiert und Meldungen konfiguriert &  \\
        \hline
        SW & Kanalfehlerdiagnose, -quittierung und –meldung programmiert &  \\
        \hline
        SW & SPS-Systemzeit lesen / schreiben programmiert &  \\
        \hline
        SW & Alle EAs aus dem Schaltplan im Programm verwendet &  \\
        \hline
        SW & Service-Funktion zum manuellen Ansteuern der Ventile vorhanden &  \\
        \hline
        SW & Service-Funktion zum Geber-/Sensorabgleich vorhanden &  \\
        \hline
        SW & Meldetexte gepflegt &  \\
        \hline
        SW & Fremdkomponenten eingebunden (z.B. Masseversorgung, Kantensteuerung, …) &  \\
        \hline
        SW & Anbindung zum Firmennetz (Auftragsdaten, Rollenprotokolle, …) fertiggestellt &  \\
        \hline
        Drive & Alle Antriebe angelegt &  \\
        \hline
        Drive & DriveCliq-Topologie korrekt &  \\
        \hline
        Drive & Regelungsart der Anwendung entsprechend ausgewählt (Servo / Vector) &  \\
        \hline
        Drive & Telegramme angelegt &  \\
        \hline
        Drive & Safety-Lizenzen vorhanden &  \\
        \hline
        Drive & Geber für die benötigte Safety-Funktion ausreichend &  \\
        \hline
        Drive & Motordaten etc. korrekt eingegeben &  \\
        \hline
        Drive & Parametrierskript ausgeführt &  \\
        \hline
        Drive & Uhrzeitsynchronisation eingerichtet / Uhrzeitmodus eingestellt &  \\
        \hline
        HMI &  &  \\
        \hline
        HMI &  &  \\
        \hline
        HMI &  &  \\
        \hline
        HMI &  &  \\
        \hline
        HMI &  &  \\
        \hline
        HMI &  &  \\
        \hline
        HMI &  &  \\
        \hline

        \caption{Projektcheckliste}\label{tab:Projektcheckliste} %\label used for referencing the figure in text
    \end{longtable}
